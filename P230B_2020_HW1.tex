\documentclass[12pt]{article}
	
	%%%%%%%%%%%%%%%%%%%%%%%%%%%%%
	%%%  THE USUAL PACKAGES  %%%%
	%%%%%%%%%%%%%%%%%%%%%%%%%%%%%
	
	\usepackage{amsmath}
	\usepackage{amssymb}
	\usepackage{amsfonts}
	\usepackage{graphicx}
	\usepackage{xcolor}
	\usepackage{nopageno}
	\usepackage{enumerate}
	\usepackage{parskip}
	
	
	\renewcommand{\thesection}{}
	\renewcommand{\thesubsection}{\arabic{subsection}}
	
	%%%%%%%%%%%%%%%%%%%%%%%%%%%%%%%%%%%%%%%%%%%%%%%
	%%%  PAGE FORMATTING and (RE)NEW COMMANDS  %%%%
	%%%%%%%%%%%%%%%%%%%%%%%%%%%%%%%%%%%%%%%%%%%%%%%
	
	\usepackage[margin=2cm]{geometry}   % reasonable margins
	
	\graphicspath{{figures/}}	        % set directory for figures
	
	% for capitalized things
	\newcommand{\acro}[1]{\textsc{\MakeLowercase{#1}}}    
	
	\numberwithin{equation}{subsection}    % set equation numbering
	\renewcommand{\tilde}{\widetilde}   % tilde over characters
	\renewcommand{\vec}[1]{\mathbf{#1}} % vectors are boldface
	
	\newcommand{\dbar}{d\mkern-6mu\mathchar'26}    % for d/2pi
	\newcommand{\ket}[1]{\left|#1\right\rangle}    % <#1|
	\newcommand{\bra}[1]{\left\langle#1\right|}    % |#1>
	\newcommand{\Xmark}{\text{\sffamily X}}        % cross out
	
	\let\olditemize\itemize
	\renewcommand{\itemize}{
	  \olditemize
	  \setlength{\itemsep}{1pt}
	  \setlength{\parskip}{0pt}
	  \setlength{\parsep}{0pt}
	}
	
	
	% Commands for temporary comments
	\newcommand{\comment}[2]{\textcolor{red}{[\textbf{#1} #2]}}
	\newcommand{\flip}[1]{{\color{red} [\textbf{Flip}: {#1}]}}
	\newcommand{\email}[1]{\texttt{\href{mailto:#1}{#1}}}
	
	\newenvironment{institutions}[1][2em]{\begin{list}{}{\setlength\leftmargin{#1}\	setlength\rightmargin{#1}}\item[]}{\end{list}}
	
	
	\usepackage{fancyhdr}		% to put preprint number
	
	
	
	
	%%%%%%%%%%%%%%%%%%%
	%%%  HYPERREF  %%%%
	%%%%%%%%%%%%%%%%%%%
	
	%% This package has to be at the end; can lead to conflicts
	\usepackage{microtype}
	\usepackage[
		colorlinks=true,
		citecolor=black,
		linkcolor=black,
		urlcolor=green!50!black,
		hypertexnames=false]{hyperref}
	




\begin{document}
\begin{center}

%%%%%%%%%%%%%%%%%%%%%%%%%%%%%%%%%%

{\Large \textsc{Weekly HW 1}:
\textbf{Generating Functions}}

%%%%%%%%%%%%%%%%%%%%%%%%%%%%%%%%%%
    
\end{center}

\vskip .4cm

\noindent
\begin{tabular*}{\textwidth}{rl}
	\textsc{Course:}& Physics 230B, \emph{Quantum Field Theory II} (2020)
	\\
	\textsc{Instructor:}& Prof. Flip Tanedo (\email{flip.tanedo@ucr.edu})
	\\
	\textsc{Due by:}& {Tuesday}, January 28
\end{tabular*}

\noindent You don't have to do them all, but you might want to try.

\vspace{1em}
{\Large \bf \sffamily Review Problems}
\vspace{-1em}

\subsection{The main result, back of the envelope style.}

Start with a Lagrangian 
\begin{align}
	L &= \frac{1}{2} x \mathcal O x + J x \ ,
\end{align}
where $\mathcal O$ is an operator. Let's be cavalier and treat $\mathcal O$ as a number. Use the Euler--Lagrange equation to derive that we may equivalently express the Lagrangian as
\begin{align}
	L &= - J \frac{1}{2\mathcal O} J \ .
\end{align}


{\footnotesize From Lancaster \& Blundell, exercise (23.1).}


\subsection{Graph Theory}

Prove the graph theory result that relates the number of loops $L$, internal lines $I$, and vertices $V$ in a graph:
\begin{align}
	L = I - V + 1.
\end{align}

\subsection{Going between metrics}

What are the momentum space Feynman propagators for a scalar field using the $(+,-,-,-)$ metric and the $(-,+,+,+)$ metric?


\subsection{Osborn's useful identity}

Prove the Osborn identity that we've been using in lecture for two function $G$ and $F$:
\begin{align}
	G(\partial_J)F(J)
	&= 
	\left.
	F(\partial_q)G(q)
	e^{Jq}
	\right|_{q=0} \ .
\end{align}
Assume that $F$ and $G$ are expandable as power series or Fourier series. Extend this to the case of many variables. 

{\footnotesize From Hugh Osborn's Part III AQFT Lent 2012 course, examples 1.}



% \vspace{1em}
\newpage
{\Large \bf \sffamily Important Ideas That Weren't in Lecture}
\vspace{-1em}

\subsection{Symmetry factors}

Remind yourself how symmetry factors work. If you don't know, then read up. The discussion in Osborn's lecture notes begins on page 16. Given a term in the potential 
\begin{align}
	V[\phi] = \frac{\lambda}{S}\phi(x)^4 \ ,
\end{align}
what is the appropriate symmetry factor, $S$, so that the Feynman rule for this vertex is $-i\lambda$?

\subsection{Zero-dimensional QFT}

Zero-dimensional QFT (quantum mechanics) is a useful system whenever things get challenging. A zero-dimensioanl model for the functional integral in QFT is
\begin{align}
	Z(\lambda) = \frac{1}{\sqrt{2\pi}} 
	\int dx\, e^{-\frac{1}{2} x^2 - \frac{\lambda}{4!}x^4} \ .
\end{align}
Assume $\lambda > 0$, as require by vacuum stability. Obtain the $N^\text{th}$ order perturbation expansion
\begin{align}
	Z_N(\lambda) 
	&= 
	\sum_{n=0}^N \left(\frac{-\lambda}{4!}\right)^n \frac{(4n)!}{2^{2n}(2n)! n!} \ .
\end{align}
Show that this gives a generating function for connected diagrams,
\begin{align}
	W(\lambda) = \log Z(\lambda) 
	= -\frac{\lambda}{8} + \frac{\lambda^2}{12} - \frac{11\lambda^3}{96} +
	\mathcal O(\lambda^4) \ .
\end{align}
Show how the coefficients in the expansion of $W(\lambda)$ are the sums of the symmetry factors of the relevant connected vacuum graphs. \text{Hint}: there is one vacuum diagram at two loops, two at three loops, and four at four loops. 


\textsc{Bonus:} Why is $Z(\lambda) < 1$? Plot $Z_N(\lambda)$ for $\lambda = 0.1$ as a function of $N$ and show where the result appears to converge to a precise factor before blowing up for larger $N$. Be expanding $e^{-x^2/2}$, obtain a convergent strong coupling expansion in terms of powers of $\lambda^{-(2n+1)/4}$ for $n=0,1,\cdots$. How many terms does one need for $\lambda=0.1$ to get the previous result?

{\footnotesize From Hugh Osborn's Part III AQFT Lent 2012 course, examples 1.}

\subsection{Time ordering}

What happened to time ordering in our correlation functions? Didn't these show up all the time during canonical quantization?


In scalar quantum field theory, for some in- and out-state spatial (fixed time) field configuration $\phi_a(\vec{x})$ and $\phi_b(\vec{x})$
\begin{align}
	\langle \phi_b(\vec{x}) | e^{-iHT} | \phi_a(\vec{x}) \rangle 
	= 
	\int \mathcal D \phi \, e^{iS}  \ ,
	\label{eq:Peskin:9.14}
\end{align}
the analog of the expression in quantum mechanics.
%
Now consider
\begin{align}
	\int \mathcal D\phi \; \phi(x_1) \phi(x_2)\, e^{iS} \ . 
	\label{eq:Peskin:2point}
\end{align}
Here $x_1$ and $x_2$ are specific points in spacetime.
The path integral is assumed to satisfy the boundary conditions
\begin{align}
	\phi(-T, \vec{x}) &= \phi_{\text{in}}(\vec{x})
	&
	\phi(T, \vec{x}) &= \phi_{\text{out}}(\vec{x}) \ ,
\end{align}
where eventually we take the `large' time $T\to \infty$.
A useful trick is to take \eqref{eq:Peskin:2point} and restrict the field $\phi$ to have a a specific spatial dependence $\phi_1(\vec{x})$ at $t=x_1^0$, and similarly $\phi_2(\vec{x})$ at $t=x_2^0$. Then integrate over the spatial configurations $\phi_{1,2}(\vec{x})$:
\begin{align}
	\int \mathcal D\phi_1(\vec{x})
	\int \mathcal D\phi_2(\vec{x})
	\int \left. \mathcal D\phi\right|_{%
	\phi(x^0_1,\vec{x})=\phi_1(\vec{x}) ,\; 
	\phi(x^0_2,\vec{x})=\phi_2(\vec{x})
	%
	} \; \phi(x_1) \phi(x_2)\, e^{iS} \ . 
	\label{eq:Peskin:2point:b}
\end{align}
This means that the $\mathcal D \phi$ integration breaks up into three separate path integrals that may be written using \eqref{eq:Peskin:9.14}. The particular amplitudes depend on the ordering of $x_1^0$ and $x_2^0$. 

Show that 
\begin{align}
	\int \mathcal D\phi \; \phi(x_1) \phi(x_2)\, e^{iS} 
	= 
	\langle \phi_\text{out} |
	e^{iHT}
	\mathcal T\left[ \phi_H(x_1) \phi_H(x_2) \right]
	e^{-iHT}
	| \phi_\text{in} \rangle \ ,
\end{align}
where $\mathcal T$ is the time ordering operation. This shows that our functional integral mumbo jumbo indeed produces the same correlation functions/amplitudes that you wrote down in QFT I.

\textsc{Hints}: Convert the states into Schr\"odinger operators using $\hat \phi_S(\vec{x}_1)|\phi_1\rangle = \phi_1(\vec{x}_1)|\phi_1 \rangle$. You should then be able to convert to absorb factors of the time evolution operator by converting from Schr\"odinger to Heisenberg fields, $\phi_H(x^0_1, \vec{x}_1) = \phi_H(x_1)$. 

{\footnotesize From Peskin \& Schroeder section 9.2.}

\subsection{Partition Function}

Banks' \emph{Modern Quantum Field Theory: A Concise Introduction} introduces the partition function as \emph{the vacuum persistence amplitude in the presence of a source},
\begin{align}
	Z[J] = \langle 0 | \mathcal T e^{i\int d^4x\, J(x)\phi(x)}|0\rangle  \ .
\end{align}
Derive this using operator methods.


{\footnotesize From Banks, equation (3.3)}

\vspace{1em}
{\Large \bf \sffamily Challenge Problems}
\vspace{-1em}

\subsection{Quantum statistical mechanics}

Do Peskin \& Schroeder problem 9.2, parts (a--d).

\vspace{1em}
{\Large \bf \sffamily 
Check your understanding
}
\vspace{-1em}

\subsection{Complex Scalar Fields}

How does everything we did for scalar field theory change if we replaced the real scalar field with a complex scalar field? Write down the action. How does the path integral formulation change compared to what we did in class? 

You can describe qualitatively, but if you do this you'd better be correct and not miss anything. I suggest re-deriving everything from scratch, using our lectures and the Osborn notes (or whatever other reference) as a guide. 

What does the free field generating function for connected diagrams, $W[J]$ look like? Or in other words, what is the analog of
\begin{align}
	Z[J] = \text{exp}\left(
	-\frac{1}{2} \int d^dx\, d^dy \, J(x) i\Delta_F(x-y)J(y)
	\right)
\end{align}
for a complex scalar field? Calculate an appropriate, non-trivial four-point correlation function using the functional methods we developed in class. Identify one trivial four-point function and explain why it's trivial.


\textsc{Comment}: if you can do this problem without having to look at your notes for too much guidance, then you're in great shape for this class. 



\end{document}